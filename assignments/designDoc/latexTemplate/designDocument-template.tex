\documentclass[12pt]{scrartcl} %or scrbook

\usepackage{xcolor}
\definecolor{darkred}{rgb}{0.75,0,0}
\definecolor{darkblue}{rgb}{0,0,0.5}
\definecolor{darkgreen}{rgb}{0,0.5,0}
\definecolor{darkergreen}{rgb}{0,0.75,0}
\definecolor{darkmagenta}{rgb}{0.55,0,0.55}
\definecolor{left}{HTML}{041832}
\definecolor{secondary}{HTML}{241024}

\usepackage[colorlinks=true,
		     urlcolor=darkblue,
		     citecolor=darkergreen,
		     linkcolor=darkblue,
		     plainpages=false,
		     pdfpagelabels]{hyperref}

\setlength{\parindent}{0pt}
\setlength{\parskip}{.25cm}

\usepackage{graphicx}

%If you want to typeset code, use minted (requires
%some additional setup)
\usepackage{minted}

%If you want to typeset algorithms, use algorithm2e:
%\usepackage[boxed,slide,linesnumbered]{algorithm2e}

\title{[Project Title]}
\subtitle{[Project Subtitle]}
\author{Joe Student\\
        \href{mailto:foo@email.com}{foo@email.com} \\
        Jane Student\\
        \href{mailto:foo@email.com}{foo@email.com} \\        
        University of Nebraska---Lincoln\\
}

\date{Fall 2525 \\
      Version 1.x
}

\begin{document}

\maketitle
\thispagestyle{empty}

\vfill

\begin{abstract}
[Provide a short abstract of this document here.  Throughout this template 
we give directions and \emph{placeholders} within square brackets on what 
the content should be.  As you update this document, the placeholders should
be removed.]
\end{abstract}

\newpage
\clearpage
\setcounter{page}{1}
\section*{Revision History}

\begin{tabular}{|l|l|l|l|}
\hline
Version & Description of Change(s) & Author(s) & Date \\
\hline
1.0 & Initial draft of this design document & Joe Student & 2525/01/01 \\
\hline
1.1 & Typo and grammar fixes & Jane Student & 2525/01/02 \\
\hline
\end{tabular}

\newpage
\tableofcontents

\newpage
\section{Introduction}

[Provide a short introduction to this document, the project and the context 
in which it is being developed.  This document needs to conform to the 
IEEE 1016 standard \cite{IEEE1016} (this is how you use citations in \LaTeX\ and BibTex).  
Do not include citations just to include them.  Only include them if they are relevant
\emph{and} you have cited them somewhere in the document.  

Keep your writing professional and write in a \emph{technical style}.  Keep things short but provide
sufficient details so that another development team would be able to reproduce
your system.  Do not use personal pronouns.  Write in a present tense \emph{as if}
the project is already done.  This document is about the \emph{design} of the system
not about the process.  Do not refer to ``we did A then we did B, etc.''  Do not
write as if this is a class project; write \emph{as if} it were a real project.

The introduction should include details on who the client is, what is their business
model, what are the major \emph{features} (not components) that the system needs to
have and support?  When you describe the components in Sections 
\ref{section:overallDesignDescription} and \ref{section:detailedDesignDescription}, 
you should justify your design decisions 
by calling back to these requirements as necessary.  If you don't include these
details, there is no way for you to justify your design decisions.]

\subsection{Purpose of this Document}

[Describe the purpose of this document; the goal(s) that its content are intended to achieve]

\subsection{Scope of the Project}

[Describe the scope of the project, what features and functionality it covers 
(at a high-level).  What does this project cover and what does the project 
\emph{not} cover (ie what is the responsibility of another team?).]

\subsection{Definitions, Acronyms, Abbreviations}

\subsubsection{Definitions}

[Define any terms that require a definition--domain specific terms, 
non-standard terms, or terms that are used in non-standard ways.  Write for
a \emph{technical audience}: do not include things that are part of the course
or that a technical audience would know already.]

\subsubsection{Abbreviations \& Acronyms}

[Define all abbreviations and acronyms used in this document here.  Only include
them if you actually use them in the document.  When used in the document for the
first time, it should include the full text.  For example, the Associate for Computing
Machinery (ACM) is a global organization of computing professionals and researchers 
that promotes the science and profession of computing.]

\begin{description}
  \item[ACM] Association for Computing Machinery
  \item[IEEE] Institute of Electrical and Electronics Engineers
  \item[UAV] Unmanned Aerial Vehicle
\end{description}
  
\section{Overall Design Description}
% this is a comment in latex
% a label can be used to make a reference to a section
\label{section:overallDesignDescription}
  
[Provide an overall \emph{high-level} and abstract description of the design.
Identify the major components, major features, etc.  Identify the technologies used
Identify and talk about the \emph{major} features of the project; this section
should be updated throughout the drafts]

\subsection{Alternative Design Options}

[If applicable, describe and discuss alternative design options that 
you considered and discuss why they were not chosen.  What advantages 
and disadvantages do the alternatives provide and what advantage/disadvantages 
do the chosen design elements provide.  Provide some justification for why the 
chosen elements? advantages/disadvantages outweighed the alternatives]

\section{Detailed Component Description}
\label{section:detailedDesignDescription}

[Provide an introduction to this section here.  Identify the next subsections
and what each one will cover.]


\subsection{Database Design}

[This section will be used to detail your database schema design (Phase III).
For your draft, you should provide a sketch of your ER diagram (which may
be replaced with a more formal one after your implementation.  Identify all tables
and their purpose; the columns in each table, etc.  Do not include a lot of 
text; your ER diagram should provide enough technical details.  Your text should
not be redundant to your diagram.  Instead, your text should \emph{justify}
the design; make reference to the introduction and to the client's business model.]

\subsubsection{Component Testing Strategy}

[This section will describe your approach to testing this particular 
component.  Describe any test cases, unit tests, or other testing 
components or artifacts that you developed for this component.  How
was test data generated (if a tool was used, this is a good opportunity
for a citation).  How many test cases did you have; how many of each type?  
\emph{Justify} why that is sufficient.  What were the outcomes of the tests?
Did the outcomes affect development or force a redesign?

You may refer to the course grader system as an external testing 
environment ``provided by the client'' or ``another QA/testing team''.]


\subsection{Class/Entity Model}

[This section should detail your classes--their state, interface and how 
they relate to each other.  Your draft should include a sketch (hand or
tool generated) of your classes using a UML diagram.  Figures and tables 
should have proper captions and be referenced in the main text just 
like in Figure \ref{figure:uav}.  Don't have one giant UML diagram; 
break it up into subfigures, collecting related classes as appropriate.
Your draft needs to provide enough detail that we can give feedback on 
your design before you submit the code for each phase.  Your sketches
should be replaced with formal diagrams in later drafts.

Identify which classes are responsible for each feature.  Classes should
follow the \emph{Single Responsibility Principle}.  This section should
be updated throughout each phase as you add more classes.]

\begin{figure}[h]%h indicates a preference to place the table here
\centering
\includegraphics[scale=1.0]{uavsUNL}
\caption{A UAV (Unmanned Aerial Vehicle) soars above Memorial Stadium.  Figures should be numbered and properly captioned.  This is just an example of how to properly include a figure.}
\label{figure:uav}
\end{figure}

\subsubsection{Component Testing Strategy}

[This section will describe your approach to testing this particular 
component.  Describe any test cases, unit tests, or other testing 
components or artifacts that you developed for this component.  How
was test data generated (if a tool was used, this is a good opportunity
for a citation).  How many test cases did you have; how many of each type?  
\emph{Justify} why that is sufficient.  What were the outcomes of the tests?
Did the outcomes affect development or force a redesign?

You may refer to the course grader system as an external testing 
environment ``provided by the client'' or ``another QA/testing team''.]

\subsection{Database Interface}

[This section will be used to detail phase IV where you modify your 
application to read from a database rather than from flat files.  
This section will detail the API that you designed--how it conformed 
to the requirements, how it worked, other tools or methods that you 
designed to assist, how it handles corner cases and the expectations 
or restrictions that you've placed on the user of the API.  What is
``good'' data and what is considered ``bad'' data and how does your
API handle it?  An example table is presented as Table 
\ref{table:assignmentPerformance}.]

\begin{table}[h] %h indicates a preference to place the table here
\centering
\caption{Average Performance on Assignments; on-time vs. late and individual vs partners.  In general, captions for Tables should appear above the table.
This is just an example of how to properly include a table.}
\label{table:assignmentPerformance}
\begin{tabular}{|l|p{1.5cm}|p{1.5cm}|p{1.5cm}|p{1.5cm}|p{1.5cm}|p{1.5cm}|p{1.5cm}|}
\hline
~ & 1 & 2 & 3 & 4 & 5 & 6 & 7 \\
\hline
On-time	& 93.16\% (78.46\%)	& 88.06\% (72.31\%)	& 87.89\% (67.69\%)	& 89.37\% (56.92\%) & 83.42\% (29.23\%) & 88.40\% (53.85\%) & 74.56\% (75.38\%) \\
\hline
Late & 88.75\% (12.31\%) & 85.28\% (20.00\%) & 70.32\% (15.38\%) & 90.40\% (15.38\%) & 82.74\% (44.62\%) & 94.22\% (15.38\%) & N/A \\
\hline
Diff & \color{red}{4.42\%} & \color{red}{2.79\%} & \color{red}{17.57\%} & \color{green}{1.03\%} & \color{red}{0.68\%} & \color{green}{5.82\%} & - \\
\hline
Individual & NA	& 88.43\% (73.85\%) & 82.32\% (33.85\%) & 87.22\% (27.69\%) & 86.40\% (23.08\%) & 82.67\% (26.15\%) & ~\\
\hline
Pairs & NA & 83.55\% (18.46\%) & 86.22\% (49.23\%) & 91.00\% (46.15\%) & 78.53\% (49.23\%) & 92.83\% (46.15\%) & ~\\
\hline
Diff & NA & \color{red}{4.88\%} & \color{green}{3.90\%} & \color{green}{3.78\%} & \color{red}{7.87\%} & \color{green}{10.16\%}	& ~\\
\hline
\end{tabular}
\end{table}

\subsubsection{Component Testing Strategy}

[This section will describe your approach to testing this particular 
component.  Describe any test cases, unit tests, or other testing 
components or artifacts that you developed for this component.  How
was test data generated (if a tool was used, this is a good opportunity
for a citation).  How many test cases did you have; how many of each type?  
\emph{Justify} why that is sufficient.  What were the outcomes of the tests?
Did the outcomes affect development or force a redesign?

You may refer to the course grader system as an external testing 
environment ``provided by the client'' or ``another QA/testing team''.]

\subsection{Design \& Integration of a Sorted List Data Structure}

[This section will be used to detail phase V where you design and 
implement a custom data structure and integrate it into your application.  
Is your list node based or array based?  What is its \emph{interface}
and how does it define a sorted list?  Is it generic?  Why?  
You can/should provide another UML diagram for this list.

You should not include large chunks of code in your document, but if you do
need to typeset code in \LaTeX\ the recommendation is the 
\texttt{minted} package.  You can typeset code inline like this: 
\mintinline{java}{List<Integer> myList} or you can define
entire blocks (usually placed within a figure environment) like in
Figure \ref{figure:helloWorldJava}.

\begin{figure}[h]%h indicates a preference to place the table here
\begin{minted}{java}
/**
 * A basic Hello World class
 */
public class HelloWorld {

  public static void main(String args[]) {
    System.out.println("Hello World");
  }
}
\end{minted}
\caption{A basic Hello World program in Java.}
\label{figure:helloWorldJava}
\end{figure}

If you do use the minted package, you may need a substantial amount of 
additional setup.  First, you need to install the \LaTeX\ \texttt{minted}
package installed \emph{and} you need to install the python pygments package
(see \url{https://pygments.org/download/}; you can use \texttt{pip install pygments}).
Once everything is installed, you must compile using the 
\mintinline{text}{--shell-escape} flag.]

\subsubsection{Component Testing Strategy}

[This section will describe your approach to testing this particular 
component.  Describe any test cases, unit tests, or other testing 
components or artifacts that you developed for this component.  How
was test data generated (if a tool was used, this is a good opportunity
for a citation).  How many test cases did you have; how many of each type?  
\emph{Justify} why that is sufficient.  What were the outcomes of the tests?
Did the outcomes affect development or force a redesign?

You may refer to the course grader system as an external testing 
environment ``provided by the client'' or ``another QA/testing team''.]

\section{Changes \& Refactoring}

[During the development lifecycle, designs and implementations may need 
to change to respond to new   requirements, fix bugs or other issues, or 
to improve earlier poor or ill-fitted designs.  Over the course of this 
project such changes and refactoring of implementations (to make them 
more efficient, more convenient, etc.) should be documented in this 
section.  If not applicable, this section may be omitted or kept as a 
placeholder with a short note indicating that no major changes or 
refactoring have been made.]

\section{Additional Material}

[This is an optional section in which you may place other materials that do not necessarily fit within the organization of the other sections.]

\addcontentsline{toc}{section}{Bibliography}
\nocite{*}
\bibliographystyle{plain}
%\bibliographystyle{elsarticle-harv}
\bibliography{bibliography}

\end{document}