\documentclass[12pt]{exam}
\usepackage{subfigure}
\usepackage{eqnarray,amsmath}

\boxedpoints
\addpoints          %Add up points across entire documentclass

\input{./assignmentPreamble}
\usetikzlibrary{decorations,decorations.pathreplacing}

\pagestyle{headandfoot}
\firstpageheader{\Large Assignment 1}{\Large CSCE 156/156H}{\Large Spring 2021}

\begin{document}

\hrule

\textbf{Instructions}
Follow instructions \emph{carefully}, failure to do so may result in
points being deducted.  You may discuss problems with your classmates
\emph{at a high level}, but all work must be your own. The CSE academic 
integrity policy is in effect (see \url{https://cse.unl.edu/academic-integrity}).
Hand in your source files via the webhandin and be sure to verify that 
your programs work with the webgrader.  

\textbf{For those in the main section}: your programs must be
written in Java, should accept command line arguments as specified, 
execute from the command line on CSE, and output properly formatted 
output to the standard output.  All your classes should be in the
\emph{default package} and your source files should have the 
following names: 

\mintinline{text}{BSA.java}, \mintinline{text}{ProteinTranslator.java}, \mintinline{text}{GameReport.java}

\textbf{For those in the honors section}: your programs must be written 
in PHP (unless you have no prior Java experience, in which case, you should
do the Java version), should accept command line arguments as specified, 
execute from the command line on CSE, and output properly formatted output to the 
standard output as specified.  Your source files should have the following 
file names: 

\mintinline{text}{bsa.php}, \mintinline{text}{proteinTranslator.php}, \mintinline{text}{gameReport.php}

\textbf{Test Cases} For each problem, you are required to submit 
two non-trivial \emph{test cases}.  A test case is an input-output 
pair that is known to be correct---the solution should be worked out 
by hand or by use of a ``known'' correct mechanism.  Simply using 
the output of your program as the valid output is \emph{not} a test 
case.  Each test case can be put into a plain text file with the 
command line argument(s) or file contents used in the input and 
well-formatted expected output.  

Name your test case files as follows:

\begin{itemize}
  \item \mintinline{text}{bsa_input_001.txt}, 
        \mintinline{text}{bsa_output_001.txt},\\
        \mintinline{text}{bsa_input_002.txt},
        \mintinline{text}{bsa_output_002.txt}
  \item \mintinline{text}{proteinTranslator_input_001.txt}, 
        \mintinline{text}{proteinTranslator_output_001.txt},\\
        \mintinline{text}{proteinTranslator_input_002.txt},
        \mintinline{text}{proteinTranslator_output_002.txt}
  \item \mintinline{text}{gameReport_input_001.txt},
        \mintinline{text}{gameReport_output_001.txt},\\
        \mintinline{text}{gameReport_input_002.txt},
        \mintinline{text}{gameReport_output_002.txt}
\end{itemize}

\begin{questions}

\newpage
\question \textbf{Program 1 -- Body Surface Area}

In several medical applications it is necessary to accurately 
estimate the \emph{Body Surface Area} (BSA) of a patient.  There are many
formulas used to estimate the area in terms of square meters ($m^2$) using
a patient's weight ($w$ in kg) and height ($h$ in cm).  Three such formulas are:
\begin{itemize}
  \item Du Bois Formula:
    $$\mathrm{BSA} = \sqrt{\frac{h \times w}{3600}}$$
  \item Haycock Formula:
    $$\mathrm{BSA} = 0.024265 \times h^{0.3964} \times w^{0.5378}$$
  \item Boyd Formula:
    $$\mathrm{BSA} = 0.0333 \times h^{0.3}  \times w^{(0.6157 - 0.00816474\ln{(w)})}$$
   where $\ln{(w)}$ is the \emph{natural log} (base $e$).  
\end{itemize}
Write a program that reads in a patient's height and weight as command 
line arguments and computes and reports estimates of the patient's 
body surface area using all three formulas.  In addition, report an average
of the three values.  For example, for a person weighing 76.5 kgs and 175.25
cm tall, the program should produce output looking \emph{something} like the following.

\begin{minted}{text}
Patient: 
  Weight: 76.5 kg
  Height: 175.25 cm
Formula     BSA (m^2)
---------   ----------
Du Bois     1.929783
Haycock     1.938204
Boyd        1.943597
Average     1.937195
\end{minted}

\begin{itemize}
  \item If using Java, your program should be runnable from the command 
    line as:\\
	\mintinline{text}{java BSA 76.5 175.25}
  \item If using PHP it should be invokable from the command line as:\\
	\mintinline{text}{php bsa.php 76.5 175.25}
\end{itemize}

\newpage
\question \textbf{Program 2 -- Protein Translation}

DNA is a molecule that encodes genetic information.  A DNA sequence is 
a string of nucleotides represented as letters A, T, C, and G (representing
the nucleobases adenine, thymine, cytosine, and guanine respectively).  
Protein sequencing in an organism consists of a two step process.  First 
the DNA is translated into RNA by replacing each thymine nucleotide with 
uracil (U).  Then, the RNA sequence is translated into a protein (a sequence
of amino acids) according to the following rules.

The RNA sequence is processed 3 bases at a time called a \emph{codon}.  
Each codon is translated into a single amino acid according to known 
encoding rules.  There are 20 such amino acids, each represented by a 
single letter in 
 $$(A,C,D,E,F,G,H,I,K,L,M,N,P,Q,R,S,T,V,W,Y)$$
Because there are $4^3 = 64$ possible codons but only 20 amino acids,
some codons translate to the same amino acid.  In addition, 3 codons
(UAA, UAG, and UGA) don't translate to proteins, but instead terminate 
the translation process (represented as the character, \mintinline{c}{x}).
Though there may be additional nucleotides left in the RNA sequence, 
they are ignored and the translation ends.  

The rules for translating trigrams are complex, but we've simplified
the process by providing some starter code that maps RNA codons (as
strings) to a protein (character).

As an example, suppose we start with the DNA sequence $AAATTCCGCGTACCC$; 
it would be encoded into RNA as $AAAUUCCGCGUACCC$; and into an amino 
acid sequence $KFRVP$.

Write a program that takes an input file name as a command line argument
which contains a DNA sequence.  The input file \emph{may} contain 
irrelevant whitespace characters to avoid very long lines.  You will 
need to \emph{ignore} any and all whitespace characters when you 
process the data.  Output the translated protein sequence to the
\emph{standard output}.  

\begin{itemize}
  \item If using Java, your program should be runnable from the command 
    line as:\\
	\mintinline{text}{java ProteinTranslator inputFile.txt}
  \item If using PHP it should be invokable from the command line as:\\
	\mintinline{text}{php proteinTranslator.php inputFile.txt}
\end{itemize}

\newpage
\question \textbf{Program 3 -- Data Projection}

Consider the CSV data below concerning video games. Each line 
contains a record of a game title, publisher, published year, 
and platform separated by commas.  However, a publisher may exist without
any game records (Ubisoft for example), and a game may exist without being
available on any platform (Contra for example).

\begin{minted}[fontsize=\scriptsize]{text}
Lego Star Wars,LucasArts,2005,Xbox
Lego Star Wars,LucasArts,2005,PlayStation 2
Lego Star Wars,LucasArts,2005,PC
Lego Star Wars,LucasArts,2005,Mac
Tie Fighter,LucasArts,1994,PC
X-Wing vs Tie Fighter,LucasArts,1997,PC
,Sony Computer Entertainment,,
,Square Enix,,
,Sega,,
Legend of Zelda,Nintendo,1987,NES
Super Mario Brothers,Nintendo,1985,NES
Zelda II: The Adventure of Link,Nintendo,1988,NES
Tetris,Nintendo,1989,Game Boy
Super Mario Land,Nintendo,1989,Game Boy
GTA IV,Rockstar Games,2008,Xbox 360
GTA IV,Rockstar Games,2008,PC
GTA 3,Rockstar Games,,
GTA 2,Rockstar Games,,
GTA,Rockstar Games,,
,Blizzard Entertainment,,
Mega Man 3,Capcom,1990,NES
Mega Man 4,Capcom,,
Contra,Capcom,,
,Atari,,
Portal,Valve,2007,PlayStation 3
Portal,Valve,2007,PC
Portal,Valve,2007,Xbox 360
Portal 2,Valve,2011,Xbox 360
Portal 2,Valve,2011,PlayStation 3
Portal 2,Valve,2011,PC
Portal 2,Valve,2011,Mac
Katamari Damacy,Namco,2004,PlayStation 2
Out of this World,Delphine Software,1992,PC
Out of this World,Delphine Software,1992,Super NES
Assassin's Creed,Ubisoft,2007,Xbox 360
,Ubisoft,,
\end{minted}

You will write a program to process such a file (whose path/name
is provided as a command line argument) and produce two reports.
\begin{itemize}
  \item A sorted list of publishers (lexicographic) along with
  	a \emph{count} of the number of games they have published.
  \item A sorted list of games (lexicographic) along with
  	a \emph{count} of the number of platforms they are available on.
\end{itemize}

For the example above, the two reports would look something like the
output below.

\begin{minted}[fontsize=\scriptsize]{text}
Publisher Game Counts
=============================
Atari                       0 
Blizzard Entertainment      0 
Capcom                      3 
Delphine Software           1 
LucasArts                   3 
Namco                       1 
Nintendo                    5 
Rockstar Games              4 
Sega                        0 
Sony Computer Entertainment 0 
Square Enix                 0 
Ubisoft                     1 
Valve                       2 

Game Platform Counts
=============================
Assassin's Creed                1 
Contra                          0 
GTA                             0 
GTA 2                           0 
GTA 3                           0 
GTA IV                          2 
Katamari Damacy                 1 
Legend of Zelda                 1 
Lego Star Wars                  4 
Mega Man 3                      1 
Mega Man 4                      0 
Out of this World               2 
Portal                          3 
Portal 2                        4 
Super Mario Brothers            1 
Super Mario Land                1 
Tetris                          1 
Tie Fighter                     1 
X-Wing vs Tie Fighter           1 
Zelda II: The Adventure of Link 1 
\end{minted}

\begin{itemize}
  \item If using Java, your program should be runnable from the command 
    line as:\\
	\mintinline{text}{java GameReport inputFile.csv}
	and output the results to the standard output.
  \item If using PHP it should be invokable from the command line as:\\
	\mintinline{text}{php gameReport.php inputFile.csv}
	and output the results to the standard output.
\end{itemize}


\end{questions}

\end{document} 