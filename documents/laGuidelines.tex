\documentclass[12pt]{scrartcl}

\usepackage[printwatermark,disablegeometry]{xwatermark}

\usepackage{epsfig,amssymb}

\usepackage{xcolor}
\usepackage{graphicx}
\usepackage{epstopdf}
\usepackage{multirow}

\definecolor{darkred}{rgb}{0.5,0,0}
\definecolor{darkgreen}{rgb}{0,0.5,0}
\usepackage{hyperref}
\hypersetup{
  letterpaper,
  colorlinks,
  linkcolor=red,
  citecolor=darkgreen,
  menucolor=darkred,
  urlcolor=blue,
  bookmarks=true,
}

\usepackage{fullpage}
\usepackage{tikz}
\pagestyle{empty} %
\usepackage{subfigure}

\definecolor{MyDarkBlue}{rgb}{0,0.08,0.45}
\definecolor{MyDarkRed}{rgb}{0.45,0.08,0}
\definecolor{MyDarkGreen}{rgb}{0.08,0.45,0.08}

\definecolor{mintedBackground}{rgb}{0.95,0.95,0.95}
\definecolor{mintedInlineBackground}{rgb}{.90,.90,1}

\usepackage[newfloat=true]{minted}

\setminted{mathescape,
           linenos,
           autogobble,
           frame=none,
           framesep=2mm,
           framerule=0.4pt,
           %label=foo,
           xleftmargin=2em,
           xrightmargin=0em,
           %startinline=true,  %PHP only, allow it to omit the PHP Tags *** with this option, variables using dollar sign in comments are treated as latex math
           numbersep=10pt, %gap between line numbers and start of line
           style=default} %syntax highlighting style, default is "default"

\setmintedinline{bgcolor={mintedBackground}}
%doesn't work with the above workaround:
\setminted{bgcolor={mintedBackground}}
\setminted[text]{bgcolor={mintedBackground},linenos=false,autogobble,xleftmargin=1em}
%\setminted[php]{bgcolor=mintedBackgroundPHP} %startinline=True}
\SetupFloatingEnvironment{listing}{name=Code Sample}
\SetupFloatingEnvironment{listing}{listname=List of Code Samples}

\setlength{\parindent}{0pt} %
\setlength{\parskip}{.25cm}
\newcommand{\comment}[1]{}

\usepackage{amsmath}
\usepackage{algorithm2e}
\SetKwInOut{Input}{input}
\SetKwInOut{Output}{output}
%NOTE: you can embed algorithms in solutions, but they cannot be floating objects; use [H] to make them non-floats

\usepackage{lastpage}

%\usepackage{titling}
\usepackage{fancyhdr}
\renewcommand*{\titlepagestyle}{fancy}
\pagestyle{fancy}
\renewcommand{\headrulewidth}{0.0pt}
\renewcommand{\footrulewidth}{0.4pt}
\lfoot{Learning Assistant Guidelines -- Computer Science II}
\cfoot{~}
\rfoot{\thepage\ / \pageref*{LastPage}}

\makeatletter
\title{Guidelines for Course Leaders \& Learning Assistants}\let\Title\@title
\subtitle{Computer Science II - Spring 2021\\
{\small
\vskip1cm
Department of Computer Science \& Engineering \\
University of Nebraska--Lincoln}
\vskip-1cm}
\date{~}
\makeatother

\begin{document}

\maketitle

%\newwatermark[allpages=true,scale=5,textmark=Draft]{},

\hrule

\section*{Overview}

The instructor sets policies in the syllabus which all students 
are expected to read, understand and adhere to. Every Course Leader
(CL) and Learning Assistant (LA) is expected to read, understand 
and also follow these policies.  Often, students will attempt to 
violate these policies or ask for special consideration.  Do not 
speculate or otherwise discuss possible exceptions to these policies.  
Direct them to the instructor and follow up with the instructor if 
necessary.

\begin{itemize}
  \item Be prepared.  Be aware of the course content and expectations.
  You are responsible for knowing the material so that you can effectively
  explain, demonstrate and guide students.  Be able to complete the assignments, 
  labs, etc.\ yourself.  If you have doubts or concerns engage with
  a Course Leader or other LA.  If you cannot resolve the issue, ask the 
  instructor for clarification.
  \item Manage your time.  You have made a commitment to this course and
  will be expected to fulfill it.  Work and plan ahead.  Be aware of 
  upcoming due dates in this course as well as your own courses, research, 
  personal obligations, etc.  Plan ahead and make appropriate accommodations 
  if you know there will be an excess of work during a period of time.
\end{itemize}

\subsection*{Course Structure}

This course is a historically large enrollment course with
multiple sections.  
Despite the size, it is our goal to foster a greater sense of 
community among these students in our department and in our 
discipline.  

In addition to a traditional lecture, we've produced dozens of 
lecture/tutorial videos for students to view before and/or after lecture. 
Lectures are livestreamed and recordings are also made available.
We have extensive required reading (mostly from my free textbook but 
also supplemental resources).  

During the weekly labs, students will be randomly paired up and 
expected to complete several peer programming exercises.  They are 
expected to complete the labs in the lab time and are graded only on 
completion.

The course is structured such that each assignment is a phase in an
overall project.  The topic and specifics change from semester to 
semester, but each has the same basic structure of iteratively
building a full database-backed application.  Each phase is due on
Fridays at midnight.  Preceding each phase, students submit a
design document draft that will be evaluated and returned to them
for improvements.  The design document serves several purposes including
giving students an early experience in technical writing, providing
an impetus for an early initial design phase, and to keep students 
on track and thinking about the project as a whole.

\subsection*{General Responsibilities}

\textbf{Learning Assistants} will have several responsibilities in 
addition to the responsibilities and expectations of the Learning 
Assistant program.

\begin{itemize}
  \item Assisting in weekly lab sessions
  \item Grading all materials (assignments and design document drafts)
  \item Mentoring and helping students in office hours and online via Piazza
  \item General administrative duties (entering grades, paperwork, etc.) as needed
  \item Other duties may include course development, materials development 
  (solution keys, future exercises, etc.) and other tasks identified by the instructor.
\end{itemize}

\textbf{Course Leaders} will have the following general responsibilities
\begin{itemize}
  \item Supervising lab sessions and assisting students in them
  \item Supervise grading and ensure that all assignments are graded 
    in a timely manner, shifting of responsibilities when issues arise, 
    and ensuring quality and consistency in grading
  \item Holding regular office hours in the designated area for the course
  \item Be in regular communication and attend weekly coordinating meetings 
  with the instructor
\end{itemize}

\section*{Communication}

\begin{itemize}
  \item Piazza is our primary means of communication, use it and encourage 
students to use it.  
  \item If you receive email or canvas notifications from students, answer 
  them but redirect them in the future to Piazza.  If the question/answer
  would be of benefit to the class as a whole, post the question/answer
  to Piazza and inform the student they can find the answer there.
  \item For communications among instructor(s), CLs and LAs, use Piazza 
  but make it a private message, viewable only to instructors or the
  individual(s) that it is intended for.
  \item If a question has been asked/answered before, link to the original
  post as your answer.
  \item Be professional in all your communications, be courteous and
  helpful.  
  \item Be prompt in answering communications.  No question or email 
  should go unanswered (or at least unacknowledged) for more than 24 business hours.\footnote{Within 24 hours but only
  on business days, i.e.\ excluding weekends and holidays}
\end{itemize}

\section*{Grading}

\subsection*{Timeline}

\begin{itemize}
  \item Assignments and design document drafts are due on Fridays at midnight.  
  Randomized grading assignments will be sent out prior to the due 
  date/time.  We will use a combination of 
  \begin{itemize}
    \item Codepost: \url{https://codepost.io} - sign in using your email used in Canvas.
    This tool is used to leave detailed comments on submitted code.
    \item The webgrader, \url{https://cse.unl.edu/~cse156/grade/grader.php} (use
    your CSE credentials to login)
  \end{itemize}
  
  \item LAs should provide extensive line-by-line comments (including
  positive comments when appropriate) through codepost.  Grade according
  to the provided rubric and in a consistent manner.  
    
  \item LAs are required to have completed their assigned grading by 
    5PM the following Tuesday (or within 48 business hours of the due 
    date).  Upon completion LAs should be available via Piazza and/or 
    email for any issues that need to be resolved.
  \item If LAs face any impediments or issues to completing
    their grading on time, they should discuss this with the instructor
    team as soon as possible.  Course Leaders may be 
    responsible for helping to resolve the issue by either 
    temporarily helping with grading or shifting grading assignments.  
  \item CLs should have everything reviewed and
    any issues resolved by 5PM the following Thursday at which time 
    grades will be released to students.

\end{itemize}

\subsection*{Directives}

\begin{itemize}
%  \item All assignment (correctness) grading is done through the online webgrader system.  
%  \item Time is limited and it should not be wasted trying to troubleshoot 
%    code that won't compile or run.  If the code is ungradeable or does not
%    compile/run then take at most 5 minutes to look over
%    the code.  If the issue can be fixed within that time frame, back up the
%    original, fix it, note the differences (via code comments) and grade 
%    accordingly.  If you cannot resolve the issue within 5 minutes, assign
%    the student a zero and move on.  This will require you to login to the
%    command line and edit the files directly.  Note that the original copy
%    stored in the webhandin system will remain.
  \item Grade in accordance to the rubric.  If the rubric 
    does not address something or there is a \emph{reasonable} uncertainty, 
    discuss it with the instruction team.
  \item Grade in a consistent manner, both between individual assignments and
    with other graders.  There should not be a significant variation in
    points deducted or awarded for similar mistakes or work.  Consistency 
    and grading quality will be checked by your CL supervisor.
  \item When you deduct points, give clear and reasonably detailed reasons
    and justifications for doing so.  Good feedback is essential for the
    students' learning experience.  Put in efforts to provide constructive
    feedback and positive feedback for good work.
  \item Be professional, positive and encouraging regardless of how well 
    the student did.  Include comments for any changes or other administrative 
    items (corrections, regrades, etc.).
  \item In general, unless otherwise stated, the formatting of output is
    left up to the student.  As long as output formatting is reasonable and
    conveys \emph{just as much} information as the expected output, it should
    be graded as correct.
\end{itemize}

\subsection*{Checklist}

The following is a list of items that should have been provided to 
the instructor and/or collected by the Learning Assistant program
as well as a checklist of administrative items necessary for all
CLs and LAs.

\begin{itemize}
  \item An appropriate photo to be used on the canvas instructors 
    page so that students can easily identify you
  \item A short bio to be included in the canvas instructors page 
  \item Your NUID
  \item Your canvas login (so as to be added to the course)
  \item Your CSE login (so as to be added to the webhandin)
  \item Your availability for each lab section so the instructor can 
    assign you to to particular lab section(s) 
  \item Your availability for weekly coordinating meetings (all Course 
    Leaders are required to attend)
  \item Establish weekly office hours (5 each week)
\end{itemize}

\section*{Vigilance}

\begin{itemize}
  \item Be on the lookout for improvements to policies, processes, 
    grading, course material, etc.  I welcome any and all feedback and would
    appreciate it.
  \item Be on the lookout for suspected academic integrity violations, odd
    code idiosyncrasies or patterns not covered in class, similarities in code, 
    disparate performance in class/lab and grades received, etc.  However,
    never confront a student directly.  Bring your concerns to the instructor and
    be sure to document everything.
  \item Be on the lookout for racial, gender, or other biases or incidents.  
    It is essential that we promote an open and equitable environment for everyone.
    If you see a potential issue or event, please intervene and correct it immediately.
    Report incidences to the instructor as soon as possible.  
\end{itemize}

\section*{Online Office Hours}

\begin{itemize}
  \item Your instructor will provide you with a ``standing meeting'' 
  zoom (\url{https://unl.zoom.us/}) link or meeting ID to use 
  \item You will need a reliable internet connection, microphone and 
  (optionally) a webcam.
  \item You should wear headphones to reduce feedback and hold your 
  office hours in a relatively quiet/noise free area
  \item When you join a meeting:
  \begin{itemize} 
    \item In the ``more'' drop down, set it so that zoom plays a chime on 
    enter/exit so that if you are running zoom in the background you are 
    made aware when a student joins
    \item Maximize the chat and ``manage participants'' windows so you can 
    see everyone and are aware of the chat contents
  \end{itemize}
  \item In general, mute and stop your video unless you need to speak
  \item It is suggested you get a physical webcam cover and use it to protect your privacy
  \item Upload an appropriate photo of yourself to your profile so students can identify you when you are not streaming video
  \item Rename your profile name to what you use with students and include an \mintinline{text}{(LA)} tag in it so students can identify that you are an LA 

  \item When a student enters, be sure to address them and ask how you can help right away.  Notify them politely if they are muted
  \item Your instructor will provide you with a \emph{host key} that allows you to claim host duties if there is no host already.
  \item As host, you will be tasked with the following additional responsibilities:
  \begin{itemize}
    \item Create breakout rooms and assign students/LAs to them
    \item Note: You \emph{may} need to establish a number of breakout rooms (create 20 and select manual assignments) if no rooms have been created as, once breakout rooms have been established, no additional rooms can be created until all have been removed
    \item Try to stay in the main room or hand off hosting duties to another LA before you go to a breakout room otherwise keep the breakout room window open so you know when others enter the main room as new students joining a lobby will only see those currently in the lobby and may assume no LAs are available
    \item Hand off host duties to another LA (if applicable) before you leave. Leaving without first reassigning host may result in a student being assigned host responsibilities
    \item Be sure to rotate LAs that you ask to help with students
  \end{itemize}
\end{itemize}


\end{document}
