%%%%%%%%%%%%%%%%%%%%%%%%%%%%%%%%%%%%%%%%%%%%%%%%%%%%%%%%%%%%%%%%%%%%%%%
%
%  Template
%  Beamer Presentation by Chris Bourke
%
%%%%%%%%%%%%%%%%%%%%%%%%%%%%%%%%%%%%%%%%%%%%%%%%%%%%%%%%%%%%%%%%%%%%%%%

\documentclass{beamer}

\usepackage{tikz}

\geometry{papersize={14cm,9cm}}

% For handout version:
\usetheme[hideothersubsections]{UNLTheme}

\usepackage{amssymb}
%\usepackage{wasysym}
\input{StandardCommands}
\definecolor{mintedBackground}{rgb}{0.95,0.95,0.95}
\definecolor{mintedInlineBackground}{rgb}{.90,.90,1}

%\usepackage{newfloat}
\usepackage{minted}
\setminted{mathescape,
               linenos,
               autogobble,
               frame=none,
               fontsize=\small,
               framesep=2mm,
               framerule=0.4pt,
               %label=foo,
               xleftmargin=2em,
               xrightmargin=0em,
               startinline=true,  %PHP only, allow it to omit the PHP Tags *** with this option, variables using dollar sign in comments are treated as latex math
               numbersep=10pt, %gap between line numbers and start of line
               style=default, %syntax highlighting style, default is "default"
               			    %gallery: http://help.farbox.com/pygments.html
			    	    %list available: pygmentize -L styles
               bgcolor=mintedBackground} %prevents breaking across pages
               
\setmintedinline{bgcolor={mintedInlineBackground}}
\setminted[text]{bgcolor={},linenos=false,autogobble,xleftmargin=1em}

\usepackage{subfig}

\title[~]{Introduction to Databases \& SQL}
\subtitle{Computer Science II -- Introduction to Computer Science II}
\author[Bourke]{Christopher M. Bourke\\ \email{cbourke@cse.unl.edu}} %
\date{~}

\begin{document}

\begin{frame}
        \titlepage
\end{frame}

\section{Introduction}

\begin{frame}
    \frametitle{Introduction to Databases \& SQL}
    \framesubtitle{}

\begin{itemize}
  \item Lifetime of a program is short-lived
  \item Applications perform small ephemeral operations
  \item Can crash and die
  \item Programs may be limited to sessions or even single requests
  \item Need a way to \emph{persist} data or program state across program lives
  \item Databases provide such a means
\end{itemize}

\end{frame}

%\begin{frame}[fragile]
%  \frametitle{Listings}
%  \framesubtitle{~}
%
%  The following code:
%\begin{minted}{sql}
%CREATE TABLE Student (
%    id primary key not null auto_increment,
%    first_name varchar(50),
%    last_name varchar(50),
%    nuid varchar(8)
%);
%\end{minted}
%\end{frame}

\section{Relational Databases}

\begin{frame}[allowframebreaks]
    \frametitle{Motivating Example}
    \framesubtitle{Flat Files}

Consider the following data, stored in a \emph{flat file}:
\begin{table}[h]
{\tiny
\begin{tabular}{|l|l|l|l|l|l|}
\hline
Course & Course Name & Student & NUID & Email\\
\hline
\hline
          & & waits, tom & 11223344 & tomwaits@hotmail.com \\
CSCE 156 & Intro to CSII & Lou Reed & 11112222 & reed@gmail.com \\
CSCE 156 & Intro to CSII & Tom Waits & 11223344 & twaits@email.com \\
CSCE 230 & Computer Hardware & Student, J.& 12345678 & jstudent@geocities.com \\
CSCE 156 & Intro to CSII & Student, John & 12345678 & jstudent@geocities.com \\
CSCE 230 & Computer Hardware & Student, J.& 12345678 & jstudent@geocities.com \\
CSCE 235 & Discrete Math & Student, John & 12345678 & jstudent@geocities.com \\
CSCE 235 & Discrete Math & Tom Waits & 11223344 & twaits@email.com \\
NONE & Null & Tom Waits & 11223344 & twaits@email.com \\
\hline
\end{tabular}
}
\caption{Course enrollment data}
\end{table}

\framebreak

Problems?
\begin{itemize}
  \item Repetition of data
  \item Incomplete data
  \item Integrity of data
  \item Organizational problems: any aggregation requires processing all records
  \item Updating information is difficult (must enumerate all possible changes, side effects so that information is not lost)
  \item Formatting Issues
  \item Concurrency Issues
\end{itemize}

\end{frame}

\begin{frame}[allowframebreaks]
    \frametitle{Relational Databases}
    \framesubtitle{Key Aspects}

Solution: Relational Database Systems (RDBS) or Relational Database Management System (RDMS)
\begin{itemize}
  \item Stores data in \emph{tables}
  \item Tables have a unique name and type description of its \emph{fields} (integer, string)
  \item Each column stores a single piece of data (field)
  \item Each row represents a record (or object!)
\framebreak
  \item Each row may have a unique \emph{primary key} which may be
  \begin{itemize}
    \item Automatically incremented
    \item An external unique identifier: SSN, ISBN, NUID
    \item Based on a combination of fields (Geographical data)
  \end{itemize}
  \item Rows in different tables are related to each other through \emph{foreign keys}
  \item Order of rows/columns is meaningless
\framebreak
  \item Supports \emph{Transactions}: an interaction or batch of interactions treated as one \emph{unit}
  \item Constraints
  \begin{itemize}
    \item Allowing or disallowing NULL
    \item Disallowing ``bad'' values (ranges)
    \item Enforcing formatting (capitalization, precision)
    \item Limiting combinations of data fields
  \end{itemize}
\framebreak
  \item ACID principles
  \begin{itemize}
    \item \textbf{Atomicity} -- Data transactions must be an all-or-nothing process
    \item Atomic operation: not divisible or decomposable
    \item \textbf{Consistency} -- Transactions will retain a state of \emph{consistency}
    \item All constraints, triggers, cascades preserve a valid state after the transaction has completed
    \item \textbf{Isolation} -- No transaction interferes or is even aware of another; state transitions are equivalent to serial transactions
    \item \textbf{Durability} -- Once committed, a transaction remains so
    \item Data is to be protected from catastrophic error (power loss/crash)
  \end{itemize}
\end{itemize}

\end{frame}

\begin{frame}
    \frametitle{Commercial RDBMs}
    \framesubtitle{~}

\begin{itemize}
  \item MS Access \onslide<2->{:)}
  \item MySQL (owned by Oracle, released under GNU GPL)
  \item PostgreSQL (true FOSS!)
  \item Informix (IBM)
  \item DB2 (IBM)
  \item SQLServer (Microsoft)
  \item Oracle Database
  \item SQLite
\end{itemize}

Others:
\begin{itemize}
  \item Google's BigTable $\rightarrow$ Spanner
  \item Apache Cassandra (Facebook)
  \item Amazon's Dynamo
\end{itemize}

\end{frame}

\begin{frame}[allowframebreaks]
    \frametitle{Advantages}

\begin{itemize}
  \item Data is \emph{structured} instead of ``just there''
  \item Better organization
  \item Duplication is minimized (with proper \emph{normalization})
  \item Updating information is easier
  \item Organization of data allows easy access
  \item Organization allows aggregation and more complex information
\framebreak
  \item Data integrity can be enforced (data types and user defined constraints)
  \item Faster
  \item Scalable
  \item Security
  \item Portability
  \item Concurrency
\end{itemize}

\end{frame}

\begin{frame}[fragile]
  \frametitle{Structured Query Language}
  \framesubtitle{~}

We interact with RDBMs using \emph{Structured Query Language} (SQL)
\begin{itemize}
  \item Common language/interface to most databases
  \item Developed by Chamberlin \& Boyce, IBM 1974
  \item Implementations may violate standards: portability issues
  \item Comments: \mintinline{sql}{--} or \mintinline{sql}{#} (MySQL on cse)
  \item Create \& manage tables: \mintinline{sql}{CREATE, ALTER, DROP}
  \item Transactions: \mintinline{sql}{START TRANSACTION, ROLLBACK, COMMIT}
\end{itemize}

\end{frame}

\begin{frame}
  \frametitle{Structured Query Language}
  \framesubtitle{CRUD}

Basic SQL functionality: \emph{CRUD}:
\begin{itemize}
  \item Create -- insert new records into existing tables
  \item Retrieve -- get a (subset) of data from specific rows/columns
  \item Update -- modify data in fields in specified rows
  \item Destroy -- delete specific rows from table(s)
\end{itemize}

\end{frame}

\begin{frame}[allowframebreaks]
  \frametitle{Misc RDMS Issues}
  \framesubtitle{~}

Important aspects that will be omitted (good advanced topics):

\textbf{Views} -- RDBSs allow you to create view of data; predefined select statements that aggregate
(or limit) data while appearing to be a separate table to the end user

\textbf{Triggers} -- SQL routines that are executed upon predefined events (inserts/updates) in order to
create side-effects on the database

\framebreak

\textbf{Stored Procedures} -- SQL routines (scripts) that are available to the end user

\textbf{Temp Tables} -- Temporary tables can be created to store intermediate values from a complex query

\textbf{Nested Queries}  -- SQL supports using subqueries to be used in other queries

\end{frame}

\section{Manipulating Tables}

\begin{frame}[fragile]
  \frametitle{MySQL}
  \framesubtitle{Getting Started}

You have access to a MySQL database on cse
\begin{itemize}
  \item Database name: your cse login
  \item Password: see the system FAQ, \url{http://cse.unl.edu/faq}
  \item Option 1: Command Line Interface (CLI):
	\mintinline{text}{>mysql -u cselogin -p}
  \item Option 2: MySQL Workbench (\url{http://cse.unl.edu/account})
\end{itemize}

Useful MySQL commands (not general SQL) to get you started:

\begin{itemize}
  \item \mintinline{sql}{USE dbdname;}
  \item \mintinline{sql}{SHOW TABLES;}
  \item \mintinline{sql}{DESCRIBE tablename;}
\end{itemize}

\end{frame}

\begin{frame}[fragile]
  \frametitle{Creating Tables}
  \framesubtitle{~}

Syntax:
\begin{minted}{sql}
CREATE TABLE table_name (
  field_name fieldType [options],
  ...
  PRIMARY KEY (keys)
);
\end{minted}

Options:
\begin{itemize}
  \item \mintinline{sql}{AUTO_INCREMENT} (for primary keys)
  \item \mintinline{sql}{NOT NULL}
  \item \mintinline{sql}{DEFAULT (value)}
\end{itemize}

\end{frame}

\begin{frame}
  \frametitle{Column Types}
  \framesubtitle{~}

\begin{itemize}
  \item \mintinline{sql}{VARCHAR(n)}  -- variable character field (or \mintinline{sql}{CHAR}, \mintinline{sql}{NCHAR}, \mintinline{sql}{NVCHAR} -- fixed size character fields)
  \item \mintinline{sql}{INTEGER} or \mintinline{sql}{INT}
  \item \mintinline{sql}{FLOAT} (or \mintinline{sql}{DOUBLE}, \mintinline{sql}{REAL}, \mintinline{sql}{DOUBLE PRECISION})
  \item \mintinline{sql}{DECIMAL(n,m)}, \mintinline{sql}{NUMERIC(n,m)} (max total digits, max decimal digits)
  \item Date/Time functions: rarely portable
  \item MySQL: see \url{http://dev.mysql.com/doc/refman/5.0/en/date-and-time-functions.html}
\end{itemize}

\end{frame}

\begin{frame}[fragile]
  \frametitle{Creating Tables}
  \framesubtitle{Example -- Old School Style}

\begin{minted}{sql}
CREATE TABLE book (
  id     INTEGER PRIMARY KEY AUTO_INCREMENT NOT NULL,
  title  VARCHAR(255) NOT NULL,
  author VARCHAR(255),
  isbn   VARCHAR(255) NOT NULL DEFAULT '',
  dewey  FLOAT,
  num_copies INTEGER DEFAULT 0
);
\end{minted}

\end{frame}

\begin{frame}[fragile]
  \frametitle{Creating Tables}
  \framesubtitle{Example -- Modern Style}

\begin{minted}{sql}
create table Book (
  bookId integer primary key auto_increment not null,
  title  varchar(255) not null,
  author varchar(255),
  isbn   varchar(255) not null default '',
  dewey  float,
  numCopies integer default 0
);
\end{minted}

\end{frame}

\begin{frame}[fragile]
  \frametitle{Primary Keys}
  \framesubtitle{~}

\begin{itemize}
  \item Records (rows) need to be distinguishable
  \item A \emph{primary key} allows us to give each record a unique identity
  \item At most one primary key per table
  \item Must be able to uniquely identify all records (not just those that exist)
  \item No two rows can have the same primary key value
  \item PKs can be one or more columns (composite key)
  \item Should not use/allow NULL values
  \item Can/should\footnote{How to handle the foreign key problem?} be automatically generated
  \item External identifiers should \emph{not} be used
  \item Should use integers, not strings or floats
\end{itemize}

\end{frame}

\begin{frame}[fragile]
  \frametitle{Keys}
  \framesubtitle{~}

\begin{itemize}
  \item Tables can have multiple keys
  \item May be a combination of columns (composite key)
  \item \mintinline{sql}{null} values are allowed
  \item Uniqueness is enforced (updates, inserts may fail)
  \item May be declared non-unique in which case it serves as an \emph{index} (allows database lookup optimization)
  \item MySQL syntax: \\
	\mintinline{sql}{KEY(column1, column2,...)}
\end{itemize}

\end{frame}

\begin{frame}[fragile]
  \frametitle{Indexes and Unique Constraints}
  \framesubtitle{~}

  \begin{itemize}
    \item The keyword \mintinline{sql}{INDEX} is often synonymous with \mintinline{sql}{KEY}
    \item Keys don't need to be unique
    \item We can make them unique by using \mintinline{sql}{UNIQUE} (as a constraint or index)
    \item Syntax: \mintinline{sql}{CONSTRAINT constraint_name UNIQUE INDEX(column_name)}
    \item Multicolumn uniqueness: \mintinline{sql}{CONSTRAINT constraint_name UNIQUE INDEX(c1, c2)}
  \end{itemize}

\end{frame}

\begin{frame}[fragile,allowframebreaks]
  \frametitle{Foreign Keys}
  \framesubtitle{~}

\begin{itemize}
  \item Relations between records in different tables can be made with \emph{foreign keys}
  \item A FK is a column that references a key (PK or regular key) in another table
  \item Inserts cannot occur if the referenced record does not exist
  \item Foreign Keys establish \emph{relationships} between tables:
   \begin{itemize}
      \item One-to-one (avoid)
      \item One-to-many relations
      \item Many-to-one
      \item Many-to-Many relations: requires a \emph{Join Table}
  \end{itemize}
\framebreak
  \item Table with FK (referencing table) references table with PK (referenced table)
  \item Deleting rows in the referenced table can be made to \emph{cascade} to the referencing records (which are deleted)
  \item Cascades can be evil
  \item MySQL Syntax:\\
	\mintinline{sql}{FOREIGN KEY (column) REFERENCES table(column)}
\end{itemize}

\end{frame}

\begin{frame}[fragile,allowframebreaks]
  \frametitle{Quick Exercise}
  \framesubtitle{~}
  
(Re)write SQL to define two tables: one for authors and one for books.
Simplify the book table, but include an ISBN as well as a constraint to
keep it unique.  Model the fact that an author may write more than one 
book.

\end{frame}

\section{Manipulating Data}

\begin{frame}[fragile]
  \frametitle{Inserting Data}
  \framesubtitle{~}

\begin{itemize}
  \item Need a way to load data into a database
  \item Numerical literals
  \item String literals: use single quote characters
  \item Ordering of columns irrelevant
  \item MySQL Syntax: \\
	\begin{minted}{sql}
	INSERT INTO table_name (c1, c2, ...) 
	  VALUES (value1, value2);
	\end{minted}
  \item Example: \\
	\begin{minted}{sql}
	INSERT INTO book (title, author, isbn) 
	  VALUES ('The Naked and the Dead', 
	          'Normal Mailer', '978-0312265052');
	\end{minted}
\end{itemize}

\end{frame}

\begin{frame}[fragile]
  \frametitle{Updating Data}
  \framesubtitle{~}

\begin{itemize}
  \item Existing data can be changed using the \mintinline{sql}{UPDATE} statement
  \item Should be used in conjunction with \emph{clauses}
  \item Syntax:\\
\begin{minted}{sql}
UPDATE table SET c1 = v1, c2 = v2, ... 
  WHERE [condition];
\end{minted}
  \item Example:
\begin{minted}{sql}
UPDATE book SET author = 'Norman Mailer' 
  WHERE isbn = '978-0312265052';
\end{minted}
\end{itemize}

\end{frame}

\begin{frame}[fragile]
  \frametitle{Deleting Data}
  \framesubtitle{~}

\begin{itemize}
  \item Data can be deleted using the \mintinline{sql}{DELETE} statement
  \item Should be used in conjunction with \emph{clauses}
  \item Unless you \emph{really} want to delete \emph{everything}
  \item Syntax:\\
	\mintinline{sql}{DELETE FROM table WHERE (condition)}
  \item Example: \\
	\mintinline{sql}{DELETE FROM book WHERE isbn = '978-0312265052';}
\end{itemize}

\end{frame}

\begin{frame}[fragile]
  \frametitle{Querying Data}
  \framesubtitle{~}

\begin{itemize}
  \item Data can be retrieved using the \mintinline{sql}{SELECT} statement
  \item Syntax:\\
	\mintinline{sql}{SELECT column1, column2... FROM table WHERE (condition);}
  \item Example:\\
\begin{minted}{sql}
SELECT author, title FROM book 
  WHERE isbn = '978-0312265052';
\end{minted}
  \item Can select \emph{all} columns by using the \mintinline{sql}{*} wildcard:\\
	\mintinline{sql}{SELECT * FROM book}
\end{itemize}

\end{frame}

\begin{frame}[fragile]
  \frametitle{Querying Data}
  \framesubtitle{Aliasing}

\begin{itemize}
  \item Names of the columns are part of the database
  \item SQL allows us to ``rename'' them in result of our query using \emph{aliasing}
  \item Syntax: \mintinline{sql}{column_name AS column_alias}
  \item Sometimes necessary if column has no name (aggregates)
\end{itemize}

\begin{minted}{sql}
SELECT title      AS bookTitle,
       num_copies AS numberOfCopies
FROM book;
\end{minted}

\end{frame}

\subsection{Clauses}

\begin{frame}[fragile]
  \frametitle{WHERE Clause}
  \framesubtitle{~}

\begin{itemize}
  \item Queries can be quantified using the \mintinline{sql}{WHERE} clause
  \item Only records matching the condition will be affected (updated, deleted, selected)
  \item Compound conditions can be composed using parentheses and:
  \begin{itemize}
    \item \mintinline{sql}{AND}
    \item \mintinline{sql}{OR}
  \end{itemize}
\end{itemize}

\begin{minted}{sql}
SELECT * FROM book 
  WHERE num_copies > 10 AND 
  (title != 'The Naked and the Dead' 
    OR author = 'Dr. Seuss');
\end{minted}

To check nullity: \mintinline{sql}{WHERE dewey IS NULL}, \mintinline{sql}{WHERE dewey IS NOT NULL}

\end{frame}

\begin{frame}[fragile]
  \frametitle{LIKE Clause}
  \framesubtitle{~}

\begin{itemize}
  \item VARCHAR values can be searched/partially matched using the \mintinline{sql}{LIKE} clause
  \item Used in conjunction with the string wildcard, \mintinline{sql}{%}
  \item Example: \\
	\mintinline{sql}{SELECT * FROM book WHERE isbn LIKE '123%';}
  \item Example: \\
	\mintinline{sql}{SELECT * FROM book WHERE author LIKE '%Mailer%';}
\end{itemize}

\end{frame}

\begin{frame}[fragile]
  \frametitle{IN Clause}
  \framesubtitle{~}

\begin{itemize}
  \item The \mintinline{sql}{IN} clause allows you to do conditionals on a \emph{set} of values
  \item Example:\\
\begin{minted}{sql}
SELECT * FROM book WHERE isbn IN ('978-0312265052', 
  '789-65486548', '681-0654895052');
\end{minted}
  \item May be used in conjunction with a nested query:
\begin{minted}{sql}
SELECT * FROM book WHERE isbn IN 
  (SELECT isbn FROM book WHERE num_copies > 10);
\end{minted}
\end{itemize}

\end{frame}

\begin{frame}[fragile]
  \frametitle{ORDER BY Clause}
  \framesubtitle{~}

\begin{itemize}
  \item In general, the order of the results of a \mintinline{sql}{SELECT} clause is irrelevant
  \item Nondeterministic, not necessarily in any order
  \item To impose an order, you can use \mintinline{sql}{ORDER BY}
  \item Can order along multiple columns
  \item Can order descending or ascending (\mintinline{sql}{DESC}, \mintinline{sql}{ASC})
  \item Example:\\
	\mintinline{sql}{SELECT * FROM book ORDER BY title;}
  \item Example:\\
	\mintinline{sql}{SELECT * FROM book ORDER BY author DESC, title ASC}
\end{itemize}

\end{frame}

\subsection{Aggregate Functions}

\begin{frame}[fragile]
  \frametitle{Aggregate Functions}
  \framesubtitle{~}

\begin{itemize}
  \item Aggregate functions allow us to compute data on the database without processing all the data in code
  \item \mintinline{sql}{COUNT} provides a mechanism to count the number of records
  \item Example:\\
	\mintinline{sql}{SELECT COUNT(*) AS numberOfTitles FROM book;}
  \item Aggregate functions: \mintinline{sql}{MAX}, \mintinline{sql}{MIN}, \mintinline{sql}{AVG}, \mintinline{sql}{SUM}
  \item Example:\\
	\mintinline{sql}{SELECT MAX(num_copies) FROM book;}
  \item Using nested queries: \\
  	\begin{minted}{sql}
	SELECT * FROM book WHERE num_copies = 
	  (SELECT MAX(num_copies) FROM book);
	\end{minted}
  \item NULL values are ignored/treated as zero
\end{itemize}

\end{frame}

\begin{frame}[fragile]
  \frametitle{GROUP BY clause}
  \framesubtitle{~}

\begin{itemize}
  \item The \mintinline{sql}{GROUP BY} clause allows you to project data with common values into a smaller set of rows
  \item Used in conjunction with aggregate functions to do more complicated aggregates
  \item Example: find total copies of all books by author:\\
	\begin{minted}{sql}
	SELECT author, SUM(num_copies) AS totalCopies 
	  FROM book GROUP BY author;
	\end{minted}
  \item The projected data can be further filtered using the \mintinline{sql}{HAVING} clause:
	\begin{minted}{sql}
	SELECT author, SUM(num_copies) AS totalCopies 
	  FROM book GROUP BY author HAVING totalCopies > 5;
	\end{minted}
  \item \mintinline{sql}{HAVING} clause evaluated \emph{after} \mintinline{sql}{GROUP BY} which is evaluated \emph{after} any \mintinline{sql}{WHERE} clause
\end{itemize}

\end{frame}

\begin{frame}[fragile,allowframebreaks]
  \frametitle{GROUP BY clause}
  \framesubtitle{Example}
{\small
Table content:
\begin{center}
\begin{tabular}{|l|l|l|}
\hline
title & author & num\_copies \\
\hline
\hline
Naked and the Dead & Norman Mailer & 10 \\
Dirk Gently's Holistic Detective Agency & Douglas Adams & 4 \\
Barbary Shore & Norman Mailer & 3 \\
The Hitchhiker's Guide to the Galaxy & Douglas Adams & 2 \\
The Long Dark Tea-Time of the Soul & Douglas Adams & 1 \\
Ender's Game & Orson Scott Card & 7 \\
\hline
\end{tabular}
\end{center}
\framebreak
Grouping: 
\begin{center}
\begin{tabular}{|l|l|l|}
\hline
title & author & num\_copies \\
\hline
\hline
Naked and the Dead & Norman Mailer & 10 \\
Barbary Shore & Norman Mailer & 3 \\
Dirk Gently's Holistic Detective Agency & Douglas Adams & 4 \\
The Hitchhiker's Guide to the Galaxy & Douglas Adams & 2 \\
The Long Dark Tea-Time of the Soul & Douglas Adams & 1 \\
Ender's Game & Orson Scott Card & 7 \\
\hline
\end{tabular}
\end{center}

Projection \& aggregation:
\begin{center}
\begin{tabular}{|l|l|l|}
\hline
author & num\_copies \\
\hline
\hline
Norman Mailer & 13 \\
Douglas Adams & 7 \\
Orson Scott Card & 7 \\
\hline
\end{tabular}
\end{center}
}
\end{frame}

\subsection{Joins}

\begin{frame}[fragile]
  \frametitle{JOINs}
  \framesubtitle{~}

A \emph{join} is a clause that combines records from two or more tables.

\begin{itemize}
  \item Result is a set of columns/rows (a ``table'')
  \item Tables are joined by shared values in specified columns
  \item Common to join via Foreign Keys
  \item Table names can be aliased for convenience
  \item Types of joins we'll look at:
  \begin{itemize}
    \item (INNER) JOIN
    \item LEFT (OUTER) JOIN
  \end{itemize}
  \item Other types of joins: Self-join, cross join (cartesian product), right outer joins, full outer joins
\end{itemize}

\end{frame}

\begin{frame}[fragile,allowframebreaks]
  \frametitle{INNER JOIN}
  \framesubtitle{~}

\begin{itemize}
  \item Most common type of join
  \item Combines rows of table A with rows of table B for all records that satisfy some predicate
  \item Predicate provided by the \mintinline{sql}{ON} clause
  \item May omit \mintinline{sql}{INNER}
  \item May provide join predicate in a \mintinline{sql}{WHERE} clause
\end{itemize}

\framebreak

\begin{minted}{sql}
SELECT * FROM book b
    INNER JOIN person p ON b.author = p.name

SELECT * FROM book b
    JOIN person p ON b.author = p.name

SELECT *
FROM book b, person p
WHERE b.author = p.name

SELECT s.student_id,s.last_name,e.address FROM student s
    JOIN email e ON s.student_id = e.student_id;
\end{minted}

\end{frame}

\begin{frame}[fragile]
  \frametitle{INNER JOIN}
  \framesubtitle{Example}

\begin{table}
\centering
{\scriptsize
\subfloat[~][Student Table]{
\begin{tabular}{|l|l|l|}
\hline
student\_id & last\_name & first\_name \\
\hline
\hline
1234 & Castro & Starlin \\
5678 & Rizzo & Anthony \\
1122 & Sveum & Dale \\
9988 & Sandberg & Ryne \\
\hline
\end{tabular}
}
\subfloat[~][Email Table]{
\begin{tabular}{|l|l|l|}
\hline
email\_id & student\_id & address \\
\hline
\hline
1111 & 9988 & rsandberg@cubbies.net \\
1112 & 9988 & rsberg@unl.edu \\
1113 & 5678 & rizzo@cubs.com \\
1114 & 1234 & number13@cubs.com\\
\hline
\end{tabular}
}
}
\end{table}

\begin{table}
\centering
{\scriptsize
\begin{tabular}{|l|l|l|}
\hline
student\_id & last\_name & adddress \\
\hline
\hline
1234 & Castro & number13@cubs.com \\
5678 & Rizzo & rizzo@cubs.com\\
9988 & Sandberg & rsandberg@cubbies.net \\
9988 & Sandberg &  rsberg@unl.edu \\
\hline 
\end{tabular}
}
\caption{Joined tables}
\end{table}

\end{frame}

\begin{frame}[fragile,allowframebreaks]
  \frametitle{LEFT OUTER JOIN}
  \framesubtitle{~}

\begin{itemize}
  \item Left Outer Join joins table A to table B, preserving all records in table A
  \item For records in A with no matching records in B: \mintinline{sql}{NULL} values used for columns in B
  \item \mintinline{sql}{OUTER} may be omitted
\end{itemize}

\framebreak

\begin{minted}{sql}
SELECT * FROM book b
    LEFT OUTER JOIN person p ON b.author = p.name

SELECT * FROM book b
    LEFT JOIN person p ON b.author = p.name

SELECT s.student_id,s.last_name,e.address FROM student s
    LEFT JOIN email e ON s.student_id = e.student_id;
\end{minted}

\framebreak

\begin{table}
\centering
{\scriptsize
\begin{tabular}{|l|l|l|}
\hline
student\_id & last\_name & adddress \\
\hline
\hline
1234 & Castro & number13@cubs.com \\
5678 & Rizzo & rizzo@cubs.com\\
9988 & Sandberg & rsandberg@cubbies.net \\
9988 & Sandberg &  rsberg@unl.edu \\
\color{blue}{1122} & \color{blue}{Sveum} & \color{blue}{NULL} \\
\hline 
\end{tabular}
}
\caption{Left-Joined Tables}
\end{table}


\end{frame}

\begin{frame}
  \frametitle{Other Joins}
  \framesubtitle{~}
  
  {\tiny Nice tutorial: \url{http://www.codeproject.com/Articles/33052/Visual-Representation-of-SQL-Joins} } 
  \begin{figure}[h]
    \includegraphics[scale=.20]{sqlJoins}
    \caption{Types of Joins}
  \end{figure}

\end{frame}

\subsection{Misc}

\begin{frame}[fragile]
  \frametitle{DISTINCT Clause}
  \framesubtitle{~}

\begin{itemize}
  \item Many records may have the same column value
  \item May want to query only the unique values
  \item May only want to count up the number of unique values
  \item SQL keyword: \mintinline{sql}{DISTINCT}
\end{itemize}

\begin{minted}{sql}
SELECT DISTINCT author FROM book;
SELECT COUNT(DISTINCT author) FROM book;
\end{minted}

\end{frame}

\section{Database Design}

\begin{frame}
  \frametitle{Database Design}
  \framesubtitle{Table Design}

  Table Design
  \begin{itemize}
    \item Identify all ``entities'' in different tables
    \item Ask ``what defines an entity'' to determine the columns and their types
    \item Properly define whether or not a column should be allowed to be null and/or defaults
    \item Each table should have a primary key
    \item Relations between tables should be identified and defined with foreign keys
    \item Security concerns: don't store sensitive information (passwords) in plaintext
  \end{itemize}

\end{frame}

\begin{frame}[allowframebreaks]
  \frametitle{Normalization}
  \framesubtitle{~}

  Normalizing a database is the process of separating data into different tables to 
  reduce or eliminate data redundancy and reduce anomolies (preserve integrity)
  when data is changed.  It also reduces the amount of book keeping necessary
  to make such changes.

  \begin{itemize}
    \item 1 Normal Form: the domain of each attribute in each table has only atomic values: each column value represents a \emph{single value}
    \begin{itemize}
      \item Example: Allowing multiple email records for one \emph{Person}
      \item Storing these as (say) a CSV in one column is a violation of 1NF
      \item Hardcoding a fixed number of columns (Email1, Email2, Email3) for multiple values is a violation of 1NF
      \item Separating records out into another table and associating them via a FK conforms to 1NF
    \end{itemize}
\framebreak
    \item 2 Normal Form: 1NF \emph{and} no non-prime attribute is dependent on any proper subset of prime attributes
    \begin{itemize}
      \item Mostly relevant for tables with composite PKs (multiple columns)
      \item If another, non-prime column is dependent only on a subset of these, its \emph{not} 2NF
      \item Must split it out groups of data into multiple tables and relate them through FKs so that non-prime column is dependent only on the subset of prime columns
      \item Example: Employee-Title-Address (must split into Employee-Title and Employee-Address)
    \end{itemize}
\framebreak
    \item 3 Normal Form: 2NF \emph{and} no non-prime column is transitively dependent on the key
    \begin{itemize}
      \item No non-prime column may depend on another non-prime column
      \item Example: Storing a price-per-unit, quantity, and total (total can be \emph{derived} from the other two)
      \item Example: CourseOfferingId-CourseId-InstructorId-Instructor Name (name should be derivable from the ID via another table)
      \item Example: OrderId-CustomerId-CustomerName (OrderId is the PK, CustomerId is FK, CustomerName should be derivable from CustomerId)
    \end{itemize}
\framebreak
    \item Bottom line: 3NF is common sense; design your database to be normalized to begin with
    \item Other advanced forms (BCNF, 4NF, 5NF, 6NF)
    \item Every non-key attribute must provide a fact about the key (1NF), the whole key (2NF), and nothing but the key (3NF), so help me Codd\footnote{Edgar Codd, inventor of the relational model, 1970--1971}
    \item Good tutorials: \url{http://support.microsoft.com/kb/283878} \\
	\url{http://phlonx.com/resources/nf3/} (available on BB)
  \end{itemize}

\end{frame}

\begin{frame}
  \frametitle{Primary Keys}
  \framesubtitle{~}

  Primary Keys
  \begin{itemize}
    \item Should be integers (not varchars nor floats)
    \item Best to allow the database to do key management (auto increment)
    \item Should \emph{not} be based on external identifiers that are not controlled by the database (NUIDs, SSNs, etc.)
    \item Be consistent in naming conventions (\mintinline{sql}{tableNameId} or \mintinline{sql}{table_name_id})
  \end{itemize}

\end{frame}

\subsection{Best Practice}

\begin{frame}[fragile]
  \frametitle{Good Practice Tip 1}
  \framesubtitle{Use consistent naming conventions}

\begin{itemize}
  \item Short, simple, descriptive names
  \item Avoid abbreviations, acronyms
  \item Use \emph{consistent} styling
  \begin{itemize}
    \item Table/field names: Lower case, underscores, singular or
    \item Camel case, pluralized
  \end{itemize}
  \item Primary key field: \mintinline{sql}{tableNameId} or \mintinline{sql}{table_name_id}
  \item Use all upper-case for SQL commands
  \item Foreign key fields should match the fields they refer to
  %\item Explicitly name foreign/primary keys (fk, pk)
  \item End goal: unambiguous, consistent, self-documenting
\end{itemize}

\end{frame}

\begin{frame}[fragile]
  \frametitle{Good Practice Tip 2}
  \framesubtitle{Ensure Good Data Integrity}

\begin{quote}
  Data can break code, code should not break data.
\end{quote}

\begin{itemize}
  \item Data/databases are a \emph{service} to code
  \item Different code, different modules can access the same data
  \item The database does \emph{not} use the code!
  \item Should do everything you can to prevent bad code from harming data (constraints, foreign \& primary keys, etc).
  \item Database is your last line of defense against bad code
\end{itemize}

\end{frame}

\begin{frame}[fragile]
  \frametitle{Good Practice Tip 3}
  \framesubtitle{Keep Business Logic Out!}

\begin{itemize}
  \item Databases offer ``programming functionality''
  \item Triggers, cascades, stored procedures, etc.
  \item \emph{Use them sparingly!!!}
  \item RDMSs are for the management and storage of data, not processing
  \item Demarcation of responsibility
  \item DBAs should not have to be Application Programmers, and vice versa
\end{itemize}

\end{frame}

\begin{frame}[fragile]
  \frametitle{Good Practice Tip 4}
  \framesubtitle{Dealing with Inheritance}

  \begin{itemize}
    \item Relational data model $\neq$ Object model
    \item Relational Databases do not have an inheritance mechanism
    \item Several strategies exist for dealing with class hierarchy with different \emph{state}
    \item Single table mapping: all state is on one table (recommended) with a \emph{discriminator column} (irrelevant columns made nullable)
    \item Joined tables: each subclass has a "sub" table joined via a foreign key from the "super" table
    \item Table-per-class: each table represents a distinct class (repeated data)
  \end{itemize}

\end{frame}

\begin{frame}[fragile]
  \frametitle{Other considerations}

  \begin{itemize}
    \item Hard vs. Soft deletes: many times it is preferable not to permanently delete data; instead an active/inactive flag is defined (care must be taken up ``new'' inserts)
  \end{itemize}

\end{frame}

\section{Current Trends}

\begin{frame}[allowframebreaks]
  \frametitle{Current Trends}

\begin{itemize}
  \item Additional Data-layer abstraction tools (JPA, ADO for .NET)
  \item XML-based databases (using XQuery)
  \item RDBMs are usually tuned to either small, frequent read/writes or large batch read-transactions
  \item Nature and scale of newer applications does not fit well with traditional RDBMs
  \item Newer applications are data intensive:
  \begin{itemize}
    \item Indexing a large number of documents
    \item High-traffic websites
    \item Large-scale delivery of multimedia (streaming video, etc.)
    \item Search applications, data mining
    \item New tools generating HUGE amounts of data (biological, chemical, sensor networks, etc)
  \end{itemize}
\framebreak
  \item Newer (revived) trend: NoSQL
  \begin{itemize}
    \item Non-relational data
    \item Sacrifices rigid ACID principles for performance
    \item Eventual consistency
    \item Limited transactions
    \item Emphasis on read-performance
    \item Simplified Key-Value data model
    \item Simple interfaces (associative arrays)
  \end{itemize}
  \item Example: Google's BigTable (key: two arbitrary string (keys) to row/column with a datetime, value: byte array)
\framebreak
  \item Even newer trend: NewSQL
  \begin{itemize}
    \item NoSQL too extreme: need scalability, but not at the cost of \emph{no} consistency or other ACID principles
    \item Large amount of data, large number of transactions, but 
    \item Short-lived, small subset of data access, repeated
    \item Sacrifice durability and concurrency
    \item Examples: Google's Spanner, new MySQL engines (MemSQL)
  \end{itemize}
\end{itemize}

\end{frame}

\section{Designing A Database}

\begin{frame}
  \frametitle{Designing A Database}
  \framesubtitle{Exercise}

\textbf{Exercise}
  Design a database to support a course roster system.  The database design should
  be able to model Students, Courses, and their relation (ability of students to \emph{enroll}
  in courses).  The system will also need to email students about updates in enrollment,
  so be sure your model is able to incorporate this functionality.

\end{frame}

\begin{frame}
  \frametitle{Designing A Database}
  \framesubtitle{~}

  \begin{figure}[h]
  \centering
  \includegraphics[scale=.35]{enrollment.png}
  \caption{A normalized database design, ER diagram generated in MySQL Workbench}
  \end{figure}

\end{frame}

\begin{frame}
  \frametitle{End Result}
  \framesubtitle{~}

\begin{itemize}
  \item Pieces of data are now organized and have a specific \emph{type}
  \item No duplication of data
  \item Entities are represented by IDs, ensuring identity (Tom Waits is now the same as t. Waits)
  \item Data integrity is enforced (only one NUID per Student)
  \item Relations are well-defined
   \begin{itemize}
     \item A student \emph{has} email(s)
     \item A course has student(s) and a student has course(s)
  \end{itemize}
  \end{itemize}
\end{frame}

\begin{frame}
  \frametitle{Data from flat file}
  \framesubtitle{~}

\input{dbFigure}

\end{frame}

\begin{frame}
  \frametitle{SQL Demo and Exercises}
  \framesubtitle{~}

\begin{figure}
\centering
\includegraphics[scale=.35]{tutorialDemos/videoGames/videoGameDatabase}
\caption{Video Game Database}
\end{figure}

Make queries as specified for the video game database described in the
figure.

\end{frame}

\begin{frame}[fragile]
  \frametitle{CSE Gotchas}
  \framesubtitle{~}

\begin{itemize}
  \item Default character set/collation may be case \emph{insensitive}
  \item \mintinline{sql}{SELECT @@character_set_database, @@collation_database;}
  \item Case sensitive: use \mintinline{sql}{latin1_general_cs} or \mintinline{sql}{utf8_bin}
  \item Default engine?  \mintinline{sql}{show engines;}
  \item Be sure to use InnoDB
\end{itemize}

Easiest workarounds:

\begin{minted}{sql}
create table Foo ( 
  ...
)Engine=InnoDB,COLLATE=latin1_general_cs;
\end{minted}

\end{frame}



\end{document}
