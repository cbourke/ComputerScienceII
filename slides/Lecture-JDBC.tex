%%%%%%%%%%%%%%%%%%%%%%%%%%%%%%%%%%%%%%%%%%%%%%%%%%%%%
%
%  Template
%  Beamer Presentation by Chris Bourke
%
%%%%%%%%%%%%%%%%%%%%%%%%%%%%%%%%%%%%%%%%%%%%%%%%%%%%%%%%%%%%%%%%%%%%%%%

\documentclass{beamer}

\geometry{papersize={14cm,9cm}}

% For handout version:
\usetheme[hideothersubsections]{UNLTheme}

\usepackage{amssymb}
\input{StandardCommands}
\definecolor{mintedBackground}{rgb}{0.95,0.95,0.95}
\definecolor{mintedInlineBackground}{rgb}{.90,.90,1}

%\usepackage{newfloat}
\usepackage{minted}
\setminted{mathescape,
               linenos,
               autogobble,
               frame=none,
               fontsize=\small,
               framesep=2mm,
               framerule=0.4pt,
               %label=foo,
               xleftmargin=2em,
               xrightmargin=0em,
               startinline=true,  %PHP only, allow it to omit the PHP Tags *** with this option, variables using dollar sign in comments are treated as latex math
               numbersep=10pt, %gap between line numbers and start of line
               style=default, %syntax highlighting style, default is "default"
               			    %gallery: http://help.farbox.com/pygments.html
			    	    %list available: pygmentize -L styles
               bgcolor=mintedBackground} %prevents breaking across pages
               
\setmintedinline{bgcolor={mintedInlineBackground}}
\setminted[text]{bgcolor={},linenos=false,autogobble,xleftmargin=1em}



\title[CSCE 156]{Introduction to Java Database Connectivity API}
\subtitle{CSCE 156 - Introduction to Computer Science II}
\author[Bourke]{Christopher M. Bourke\\ \email{cbourke@cse.unl.edu}} %
\date{~}

\begin{document}

\begin{frame}
        \titlepage
\end{frame}

\section{Introduction}

\begin{frame}[fragile]
  \frametitle{Java Database Connectivity API}
  \framesubtitle{~}

Java Database Connectivity (JDBC)

\begin{itemize}
  \item General API (using interfaces) for Java client code to connect/interact with a database
  \item Database providers (IBM, Oracle, etc.) provide \emph{drivers}
  \item Driver: specific implementation of the API for interacting with a particular database 
  \item Support for 
  \begin{itemize}
    \item \mintinline{java}{Connection}
    \item \mintinline{java}{PreparedStatement}
    \item \mintinline{java}{ResultSet}
    \item Common Java data types (\mintinline{java}{Integer, Double, String})
  \end{itemize}
\end{itemize}

\end{frame}

\section{Step-by-step}

\begin{frame}[fragile]
  \frametitle{JDBC: basic step-by-step}
  \framesubtitle{~}

\begin{enumerate}
  \item Load the database JDBC driver\\
	Note: your particular driver (.jar file) must be in the class or build path of your project
  \item Make a connection to the database
  \item Formulate your query(ies) \& prepare your statement (set parameters)
  \item Execute your query
  \item If its a \mintinline{java}{SELECT} query: 
    \begin{enumerate}
      \item Get your \emph{result set}
      \item Process your results
    \end{enumerate}
  \item Clean up your resources (close resources, close connection)
\end{enumerate}

\end{frame}

\begin{frame}[fragile]
  \frametitle{JDBC}
  \framesubtitle{Reflectively loading a driver}

\begin{itemize}
  \item For portability, applications written toward JDBC API, not a particular driver
  \item Driver is loaded at run time through reflection
  \item Could be made configurable or delegated by some controller
\end{itemize}

\begin{minted}{java}
try {
    Class.forName("com.mysql.jdbc.Driver").newInstance();
} catch (InstantiationException e) {
    ...
} catch (IllegalAccessException e) {
    ...
} catch (ClassNotFoundException e) {
    ...
}
\end{minted}

\end{frame}

\begin{frame}[fragile]
  \frametitle{JDBC}
  \framesubtitle{Connection}

Java provides connectivity through \mintinline{java}{java.sql.Connection}:

\begin{minted}{java}
String url = "jdbc:mysql://cse.unl.edu/cselogin";
String u = "cselogin";
String p = "mysqlpasswd";
Connection conn = null;
try {
  conn = DriverManager.getConnection(url, u, p);
} catch (SQLException sqle) {
  ...
}
\end{minted}

\end{frame}

\begin{frame}[fragile]
  \frametitle{JDBC}
  \framesubtitle{Transactions}

\begin{itemize}
  \item By default, all queries are auto-commit
  \item To change this, use \mintinline{java}{conn.setAutoCommit(false)}
  \item No changes committed until \mintinline{java}{conn.commit()} is called
  \item Implicitly: new transaction after each commit
  \item Able to explicitly rollback using \mintinline{java}{conn.rollback()}
  \item Some drivers may also support \mintinline{java}{conn.setReadOnly(true)}
\end{itemize}

\end{frame}

\begin{frame}[fragile,allowframebreaks]
  \frametitle{JDBC}
  \framesubtitle{Querying -- Prepared Statement}

\begin{itemize}
  \item Always good to use \mintinline{java}{PreparedStatement}
  \item Can define \emph{parameters} using \mintinline{java}{?}
  \item Parameters indexed by \mintinline{java}{1..n}
  \item Can be reused (parameters reset and required)
  \item Parameters are \emph{safe}!
  \item Special characters are \emph{escaped} 
  \item Potentially unsafe SQL code is \emph{sanitized}	
\end{itemize}

\framebreak

\begin{minted}{java}
String query = "SELECT last_name AS lastName " + 
               "FROM user WHERE nuid = ?";
PreparedStatement ps = null;
try {
  ps = conn.prepareStatement(query);
  ps.setString(1, "35140602");
} catch (SQLException sqle) {
  ...
}
\end{minted}

\end{frame}

\begin{frame}[fragile,allowframebreaks]
  \frametitle{JDBC}
  \framesubtitle{Querying -- Result Sets}

\begin{itemize}
  \item \mintinline{java}{executeQuery()} is for read-only (select statements)
  \item Select statements return \emph{results}: columns and rows
  \item Results are encapsulated in a Java \mintinline{java}{ResultSet} object
  \item Initially a result set ``points'' just before the first \emph{row}
  \item Iterating through a \mintinline{java}{ResultSet}: \mintinline{java}{rs.next()} 
  \item Returns a boolean: true if the iteration was successful, false otherwise
  \item If successful, the ``current'' result row is now pointed to
  \item Columns can be referenced by name (or alias) using a \mintinline{java}{String} or
  \item Columns can be accessed via index ($1, \ldots, $)
  \item Standard getters provide functionality to get-and-cast columns
\end{itemize}

\begin{minted}{java}
ResultSet rs = null;
try {
  rs = ps.executeQuery();
  while(rs.next()) {
    Integer nuid = rs.getInt("nuid")
    String firstName = rs.getString("firstName");
  }
} catch (SQLException sqle) {
  ...
}
\end{minted}

\end{frame}

\begin{frame}[fragile]
  \frametitle{JDBC}
  \framesubtitle{Querying -- Updates}

\begin{itemize}
  \item \emph{Always} use a prepared statement!
  \item Same syntax holds for INSERT statements
\end{itemize}

\begin{minted}{java}
String query = "UPDATE user SET email = ?, " +
               "last_updated = ? WHERE nuid = ?";
PreparedStatement ps = null;
try {
  ps = conn.prepareStatement(query);
  ps.setString(1, "cmbourke@gmail.com");
  ps.setString(2, "2011-01-01 00:00:01");
  ps.setString(3, "35140602");
  ps.executeUpdate();
} catch (SQLException sqle) {
  ...
}
\end{minted}

\end{frame}

\begin{frame}[fragile]
  \frametitle{JDBC}
  \framesubtitle{Good Practices -- Rethrow Exceptions}

\begin{itemize}
  \item Most methods explicitly throw \mintinline{java}{SQLException} 
  \item This is a \emph{checked exception} that \emph{must} be caught and handled
  \item Occurs with DB errors or program bugs
  \item Little can be done either way
  \item Good to catch, log and rethrow 
  \item Even better: use a logging utility like log4j 
\end{itemize}

\begin{minted}{java}
...
} catch (SQLException sqle) {
    System.out.println("SQLException: ");
    sqle.printStackTrace();
    throw new RuntimeException(sqle);
}
\end{minted}

\end{frame}

\begin{frame}[fragile]
  \frametitle{JDBC}
  \framesubtitle{Cleaning Up}

\begin{itemize}
  \item Objects hold onto valuable external resources
  \item Network traffic (keep alive), limited connection pool, etc.
  \item Best practice to release resources as soon as they are no longer needed: \mintinline{java}{close()} method
\end{itemize}

\begin{minted}{java}
try {
  if(rs != null && !rs.isClosed())
    rs.close();
  if(ps != null && !ps.isClosed())
    ps.close();
  if(conn != null && !conn.isClosed())
    conn.close();
} catch (SQLException e) {
  ...
}
\end{minted}

\end{frame}

\begin{frame}[fragile]
  \frametitle{JDBC}
  \framesubtitle{Full Example Demonstration}

A full demonstration is available in the \mintinline{java}{unl.cse.jdbc} package in the SVN.

Demonstration based on the student/course enrollment database.

\end{frame}

\section{Good Practices}

\begin{frame}[fragile]
  \frametitle{Good Practice Tip 1}
  \framesubtitle{ALWAYS use Prepared Statements}

When available, in any framework or language, always use prepared statements

\begin{itemize}
  \item Safer
  \item Better for batch queries
  \item Myth: no performance hit
  \item Protects against \emph{injection attacks}
  \item Using just one method: more uniform, less of a chance of a mistake
  \item Unfortunately: some frameworks support \emph{named parameters}, not JDBC
\end{itemize}

\end{frame}

\begin{frame}[fragile]
  \frametitle{Injection Attack}
  \framesubtitle{Example}

\begin{itemize}
  \item Say we pull a string value from a web form (\mintinline{java}{lastName})
  \item Not using a prepared statement:\\
\begin{minted}{java}
String query = "SELECT primary_email FROM user " + 
               "WHERE last_name = '"+lastName+"'";
\end{minted}               
  \item Without scrubbing the input, say a user enters:\\
	\mintinline{text}{a';DROP TABLE user;}
  \item Actual query run:\\
	\mintinline[fontsize=\footnotesize]{text}{SELECT primary_email FROM user WHERE last = 'a';DROP TABLE users;}
  \item Another example: input: \mintinline{text}{"' OR '1'='1"}
  \item Actual query:\\
	\mintinline[fontsize=\footnotesize]{text}{SELECT primary_email FROM user WHERE last_name = '' OR '1'='1'}
  \item In detail: \url{https://www.netsparker.com/blog/web-security/sql-injection-cheat-sheet/}	
\end{itemize}

\end{frame}

\begin{frame}[fragile]
  \frametitle{Injection Attack}
  \framesubtitle{Example}

\begin{figure}
\centering
\includegraphics[scale=.5]{exploits_of_a_mom}
\end{figure}

\end{frame}

\begin{frame}[fragile]
  \frametitle{Good Practice Tip 2}
  \framesubtitle{Enumerate fields in SELECT statements}

\begin{itemize}
  \item Using \mintinline{java}{SELECT * ...} grabs all fields even if you don't use them
  \item Be intentional about what you want/need, only the minimal set
  \item Allows the database to optimize, reduces network traffic
  \item Protects against table changes
  \item Use aliasing (\mintinline{java}{first_name AS firstName}) on all fields to reduce affects of changes to field names
\end{itemize}

\end{frame}

\begin{frame}[fragile]
  \frametitle{Additional Issues}
  \framesubtitle{~}

Additional Issues
\begin{itemize}
  \item Security Issues
  \begin{itemize}
    \item Where/how should database passwords be stored?
    \item Good security policy: \emph{assume} an attacker has your password \& 
	  take the necessary precautions (secure the server and network)
    \item Do not store sensitive data unencrypted
  \end{itemize}
  \item Efficiency Issues
  \begin{itemize}
    \item \alert{Repeat}: \textbf{close your resources}
    \item Connection Pools
    \item Good normalization, design, \& practice
  \end{itemize}
\end{itemize}

\end{frame}


\section{Resources}

\begin{frame}
  \frametitle{Resources}
  \framesubtitle{~}

\begin{itemize}
  \item MySQL 5.1 Reference Manual (\url{http://dev.mysql.com/doc/refman/5.1/en/index.html})
  \item MySQL Community Server (\url{http://www.mysql.com/downloads/})
  \item MySQL Workbench -- a MySQL GUI (\url{http://wb.mysql.com/})
  \item Connector/J -- MySQL JDBC connector (\url{http://www.mysql.com/downloads/connector/j/})
  \item Stanford's \emph{Introduction to Databases} free online course: \url{http://db-class.com/}
\end{itemize}

\end{frame}

\begin{frame}[fragile,allowframebreaks]
  \frametitle{Log4j}
  \framesubtitle{~}

  \begin{itemize}
    \item Home: \url{http://logging.apache.org/}
    \item Standard output is not appropriate for deployed applications (may not exist, certainly no one is ``listening'' to it)
    \item Logging provides valuable debugging, metrics, and auditing information
    \item Provides runtime configuration, formatting, rolling files, etc.
    \framebreak
    \item Supports granularity in different levels of logging (\mintinline{java}{ERROR} to \mintinline{java}{DEBUG})
    \item Usage: give each loggable class  a static logger:\\
\begin{minted}{java}
private static org.apache.log4j.Logger log = 
  Logger.getLogger(MyClass.class);
\end{minted}  
    \item Then use it: \mintinline{java}{log.error("SQLException: ", e);}
    \item Configure using a \mintinline{java}{log4j.properties} file (must be in the class path)
    \item Or: call \mintinline{java}{BasicConfigurator.configure();} in your \mintinline{java}{main} to have a default setup
  \end{itemize}

\end{frame}



\begin{frame}
  \frametitle{Exercise}
  \framesubtitle{~}

  \begin{quote}
    Write basic CRUD methods for the \mintinline{java}{Employee}/\mintinline{java}{Person} tables by writing
    static methods to insert, delete, retrieve and update records in both tables.
  \end{quote}

\end{frame}

\end{document}